
Astronomy has entered the big data era with many projects creating petabytes of data per year. This data is often processed by complex multi-step pipelines consisting of various algorithms. Understanding the scalability of astronomical algorithms theoretically, in a controlled environment, and in production is important for making predictions for the data reduction of future projects and upcoming telescopes. 

The Low Frequency Array (LOFAR) \citep{LOFAR} is a leading European low-frequency radio telescope. The majority of LOFAR's stations are in the Netherlands, however the telescope can use stations across Europe to create ultra-high resolution radio maps. LOFAR data needs to undergo several computationally intensive processing steps before obtaining a final scientific image. 

To create a broadband image, LOFAR data is first processed by a Direction Independent (DI) Calibration pipeline followed by Direction Dependent (DD) Calibration pipeline \citep[e.g.][]{lofar_prefactor, Wendy_bootes,tassesmirnov, tasse2018faceting}. The goal of DI calibration is to remove effects that are constant across the target field such as radio frequency interference, contamination by bright off-axis sources and antenna gains. After this step, DD Calibration focuses on removing effects which vary across the field, such as ionospheric and beam effects. The result of these two pipelines is a science-ready image. 

Our implementation of the DI LOFAR processing, \texttt{prefactor}, can be parallelized on a high throughput cluster \citep{mechev17}. The Direction Dependent processing, implemented in \texttt{ddf-pipeline}\footnote{Available at \raggedright\href{https://github.com/mhardcastle/ddf-pipeline/releases}{https://github.com/mhardcastle/ddf-pipeline/releases}}, is subsequently performed on a single HPC node. 

The LOFAR Surveys Key Science Project (SKSP) \citep{lotss, LOTSS_DR2} is a long running project consisting of several low frequency surveys of the northern sky. The broadest tier of the survey, LoTSS, will use more than 3000 8-hour observations to create maps with a noise levels below 100 $\mu$Jy. We have already processed more than 500 of these observations using the \texttt{prefactor} DI pipeline \citep{lofar_prefactor, prefactor_zenodo}. 

While the current LoTSS imaging algorithms can process data averaged by up to a factor of 64 in frequency and time, it is important to understand how LOFAR processing scales with processing parameters, such as averaging parameters. Since LOFAR data is used by multiple scientific teams, not every team can produce scientific results from data averaged by such a high factor. Users from those teams need to be able to predict the time and computational resources required to process their data, taking into account the increasing LOFAR observation rates, data sizes and scientific requirements. 

We study the scalability of processing LOFAR data, by setting up processing of a sample SKSP data set on an isolated node on the \texttt{GINA} cluster at SURFsara, part of the Dutch national e-infrastructure \citep{dutch_einfra}. We test the software performance as a function of several parameters, including averaging parameters, number of CPUs and calibration model size. Additionally, we test the performance of the underlying infrastructure, i.e. queuing  and download time, for the same parameters. Finally, we compare those isolated tests with our production runs of the \texttt{prefactor} pipeline to measure the overhead incurred by running on a shared system. 

We discover that the computationally intensive LOFAR processing steps scale linearly with data size, and calibration model size. Additionally, we find that the time taken by these steps is inversely proportional to the number of CPUs used. We discover that the time to download and extract data on the \texttt{GINA} cluster is linear with size up to 32GB, but becomes slower beyond this data size. We also find that the queuing time on the \texttt{GINA} cluster grows exponentially for jobs requesting more than 8 CPUs. We validate these isolated tests with production runs of LOFAR data from the past six months. We combine all these tests into a single model and show its prediction power by testing the processing time for different combinations of parameters. Finally, we discuss the utility of our method, the results in this work and applications to the SKSP projects, the broader impact of our results to LOFAR processing and the applications for other large astronomical surveys. The major contributions of this work can be summarized as:

\begin{itemize}
    \item A model of processing time for the  LOFAR Direction Independent Calibration Pipeline.
    \item A model of queuing time and file transfer time which is used by current or future jobs processed on the \texttt{GINA} cluster.
    \item Quantification of overheads incurred when processing in production. 
    \item Validation of our methods with discussion of future applications. 
\end{itemize}

We introduce LOFAR processing and other related work in Section \ref{sec:ch6_related} and describe our software setup and data processing methods in Section \ref{sec:ch6_methods}. We present our results and performance model in Section \ref{sec:ch6_results} and discussions and conclusions in Section \ref{sec:ch6_discussions}. 

