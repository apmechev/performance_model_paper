% \url{http://authors.aps.org/revtex4/}.

% When commands are referred to in this example file, they are always
% shown with their required arguments, using normal \TeX{} format. In
% this format, \verb+#1+, \verb+#2+, etc. stand for required
% author-supplied arguments to commands. For example, in
% \verb+\section{#1}+ the \verb+#1+ stands for the title text of the
% author's section heading, and in \verb+\title{#1}+ the \verb+#1+
% stands for the title text of the paper.

% Line breaks in section headings at all levels can be introduced using
% \textbackslash\textbackslash. A blank input line tells \TeX\ that the
% paragraph has ended. Note that top-level section headings are
% automatically uppercased. If a specific letter or word should appear in
% lowercase instead, you must escape it using \verb+\lowercase{#1}+ as
% in the word ``via'' above.



% This file may be formatted in either the \texttt{preprint} or
% \texttt{reprint} style. \texttt{reprint} format mimics final journal output. 
% Either format may be used for submission purposes. \texttt{letter} sized paper should
% be used when submitting to APS journals.

% \subsubsection{Wide text (A level-3 head)}
% The \texttt{widetext} environment will make the text the width of the
% full page, as on page~\pageref{eq:wideeq}. (Note the use the
% \verb+\pageref{#1}+ command to refer to the page number.) 
% \paragraph{Note (Fourth-level head is run in)}
% The width-changing commands only take effect in two-column formatting. 
% There is no effect if text is in a single column.

% \subsection{\label{sec:citeref}Citations and References}
% A citation in text uses the command \verb+\cite{#1}+ or
% \verb+\onlinecite{#1}+ and refers to an entry in the bibliography. 
% An entry in the bibliography is a reference to another document.

% \subsubsection{Citations}
% Because REV\TeX\ uses the \verb+natbib+ package of Patrick Daly, 
% the entire repertoire of commands in that package are available for your document;
% see the \verb+natbib+ documentation for further details. Please note that
% REV\TeX\ requires version 8.31a or later of \verb+natbib+.

% \paragraph{Syntax}
% The argument of \verb+\cite+ may be a single \emph{key}, 
% or may consist of a comma-separated list of keys.
% The citation \emph{key} may contain 
% letters, numbers, the dash (-) character, or the period (.) character. 
% New with natbib 8.3 is an extension to the syntax that allows for 
% a star (*) form and two optional arguments on the citation key itself.
% The syntax of the \verb+\cite+ command is thus (informally stated)
% \begin{quotation}\flushleft\leftskip1em
% \verb+\cite+ \verb+{+ \emph{key} \verb+}+, or\\
% \verb+\cite+ \verb+{+ \emph{optarg+key} \verb+}+, or\\
% \verb+\cite+ \verb+{+ \emph{optarg+key} \verb+,+ \emph{optarg+key}\ldots \verb+}+,
% \end{quotation}\noindent
% where \emph{optarg+key} signifies 
% \begin{quotation}\flushleft\leftskip1em
% \emph{key}, or\\
% \texttt{*}\emph{key}, or\\
% \texttt{[}\emph{pre}\texttt{]}\emph{key}, or\\
% \texttt{[}\emph{pre}\texttt{]}\texttt{[}\emph{post}\texttt{]}\emph{key}, or even\\
% \texttt{*}\texttt{[}\emph{pre}\texttt{]}\texttt{[}\emph{post}\texttt{]}\emph{key}.
% \end{quotation}\noindent
% where \emph{pre} and \emph{post} is whatever text you wish to place 
% at the beginning and end, respectively, of the bibliographic reference
% (see Ref.~[\onlinecite{witten2001}] and the two under Ref.~[\onlinecite{feyn54}]).
% (Keep in mind that no automatic space or punctuation is applied.)
% It is highly recommended that you put the entire \emph{pre} or \emph{post} portion 
% within its own set of braces, for example: 
% \verb+\cite+ \verb+{+ \texttt{[} \verb+{+\emph{text}\verb+}+\texttt{]}\emph{key}\verb+}+.
% The extra set of braces will keep \LaTeX\ out of trouble if your \emph{text} contains the comma (,) character.

% The star (*) modifier to the \emph{key} signifies that the reference is to be 
% merged with the previous reference into a single bibliographic entry, 
% a common idiom in APS and AIP articles (see below, Ref.~[\onlinecite{epr}]). 
% When references are merged in this way, they are separated b y a semicolon instead of 
% the period (full stop) that would otherwise appear.

% \paragraph{Eliding repeated information}
% When a reference is merged, some of its fields may be elided: for example, 
% when the author matches that of the previous reference, it is omitted. 
% If both author and journal match, both are omitted.
% If the journal matches, but the author does not, the journal is replaced by \emph{ibid.},
% as exemplified by Ref.~[\onlinecite{epr}]. 
% These rules embody common editorial practice in APS and AIP journals and will only
% be in effect if the markup features of the APS and AIP Bib\TeX\ styles is employed.

% \paragraph{The options of the cite command itself}
% Please note that optional arguments to the \emph{key} change the reference in the bibliography, 
% not the citation in the body of the document. 
% For the latter, use the optional arguments of the \verb+\cite+ command itself:
% \verb+\cite+ \texttt{*}\allowbreak
% \texttt{[}\emph{pre-cite}\texttt{]}\allowbreak
% \texttt{[}\emph{post-cite}\texttt{]}\allowbreak
% \verb+{+\emph{key-list}\verb+}+.
